% vim: set ts=4 sw=4 ai et tw=74:

%\addcontentsline{toc}{section}{Desenho do Sistema}

%dizer que modelos importamos e em que formatos
%%dizer porque e que escolhemos esses formatos
%codigo

Carregar objectos no mundo, ou melhor, modelos para objectos, foi um dos grandes desafios. Tanto pela limitação da disponibilidade gratuita, como pelas limitações do carregamento dos modelos, do tipo \textit{obj}, que nos foram inicialmente apresentados.

Modelos \textit{obj} não são carregados com textura de origem, ou seja, o uso de texturas implicaria a implementação da funcionalidade. Em acréscimo nas limtações está a ausência de animações.
Neste contexto surgiu \textit{md2}, um formato original do motor do jogo quake.

\textit{MD2} é um formato abundante e bastante frequente existindo ainda uma ferramenta que facilita o carregamento para opengl, o {\bf md2loader}.

No jogo são carregados sobre a forma de modelos \textit{md2} os seguintes objectos:
\begin{itemize}
\item Player;
\item Torres;
\item Chaves;
\item Edifício do Tesouro.
\end{itemize}

