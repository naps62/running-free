O input é tratado através da classe InputManager. Esta classe mantém uma estrutura com o estado actual de todas as teclas utilizadas no jogo, bem como informação sobre as coordenadas e movimento do rato. Os eventos de pressionar ou libertar uma tecla apenas accionam um método que altera o estado da tecla correspondente na estrutura.

Assim, em qualquer ponto do código, é possivel saber o estado de uma tecla (pressionada ou não), e consoante essa flag, executar uma determinada acção. Esta estruturação permite uma gestão mais flexivel do input do que aquela utilizada nas aulas práticas, em que a função de gestão do input executava as próprias acções. Isto fazia com que a interacção fosse mais limitada, não sendo considerada a utilização de várias teclas ao mesmo tempo. Essa funcionalidade já é obtida por uma estrutura como a classe aqui utilizada.
