% 
%%terreno plano v
%%dois tipos de camara v
%%aleatoriedade e radar v
%%modelos importados e formato v
%%uma unidade em OpenGL corresponde a 0.1metros v

Existem dois modos de câmara: \textit{Third Person Shooter} (tps) e outra \textit{First Person Shooter} (fps). A mudança entre estes dois modos é feita através de uma tecla pré-definida.

O radar não indica a direcção em que está a chave e tem um raio de acção limitado a 500 metros. Este é apresentado no canto inferior esquerdo em modo textual.

O jogo pode correr em modo full screen e em modo janela, mas nao é possivel alternar no decorrer do jogo, para tal tem de se alterar a variável no ficheiro \textit{config.ini}, de forma a executar em full screen.

Grande parte das configurações relevantes do jogo, como velocidade do jogador, numero de torres, tempo entre disparos, tamanho do mapa, configurações de iluminação, deslocamento consoante o movimento do rato, resources para modelos, musica, range das torres, entre muitos outros, estão contido no documento \textit{config.ini}, documento este que será explicado mais à frente.

Por último são necessário alguns modelos para as torres, chaves, herói, etc. Por opção, estes modelos são do formato \textit{.md2}. Adiante será desenvolvido este ponto e explicada esta escolha.
