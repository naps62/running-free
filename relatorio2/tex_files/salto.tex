Um extra que foi relativamente fácil de implementar foi permitir que o jogador salte. Isto é um aspecto que acaba por encaixar naturalmente no jogo, por ser intuitiva a intenção de saltar em jogos dentro deste estilo. Não foi demasiado complexo e adiciona um novo elemento de jogabilidade que enriquece bastante o jogo, e possibilita uma forma alternativa para o jogador se desviar das balas.
Isto foi feito com recurso ao seguinte código acrescentado à função de update do jogador:

\begin{lstlisting}
if (space == KEY_ON && !isJumping && canJump) {
		isJumping = true;
		jump_time = 0;
		Sound::play(SOUND_JUMP);
	}

	if (isJumping) {
		jump_time++;
		//coords->y += jumpOff(jump_time) - jumpOff(jump_time - 1);
		coords->y += jumpOff(jump_time);
		if (anim->get_anim() != MOVE_JUMP)
			anim->set_anim(MOVE_JUMP);

		float max_height = g_map->triangulateHeight(coords->x, coords->z);
		if (coords->y < max_height) {
			coords->y = max_height;
			isJumping = false;
			anim->set_anim(MOVE_NONE);
			canJump = false;
			glutTimerFunc(jump_cooldown, GLManager::allowPlayerJump, 0);
		}

	} else if (isMoving()) {
		coords->y = g_map->triangulateHeight(coords->x, coords->z);
		if (anim->get_anim() != MOVE_WALK)
			anim->set_anim(MOVE_WALK);

	} else {
		anim->set_anim(MOVE_NONE);
	}
\end{lstlisting}

A função de salto é a seguinte:

\begin{lstlisting}
float Player::jumpOff(int off) {
	return -0.1 * off + 2;
}
\end{lstlisting}

Esta função corresponde à derivada da função de salto idealizada para o jogador. Assim para cada instante sabe-se o incremento a dar à altura do jogador durante o tempo em que este está a saltar.
