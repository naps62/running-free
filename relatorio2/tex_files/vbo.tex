Os Vertex Buffer Objects (VBOs) disponibilizam uma método de render mais eficiente para um conjunto de vértices estático. Através desta ferramenta é possível definir, na fase de inicialização do programa, um conjunto de vértices com todas as suas propriedades (coordenadas de textura, cores, normais) e armazená-los previamente num buffer na placa gráfica.

Neste projecto foram usados VBOs para o desenho do mapa de jogo, por ser o objecto que é estático ao longo de todo o decorrer do jogo. A sua implementação foi feita à custa da função Map::initVBO.

\begin{lstlisting}[caption=Cálculo dos buffers para VBOs (alocação)]
void Map::initVBO() {
	//float *vertexB, *texB, *normalB;
	int vertexSize = grid_n * grid_n * 3 * sizeof (float);
	int texSize = grid_n * grid_n * 2 * sizeof (float);
	int normalSize = grid_n * grid_n * 3 * sizeof (float);
	int stripSize = grid_n * 2 * sizeof (unsigned int);
	n_strips = grid_n - 1;

	//cria espaco temporario para preencher os buffers
	vertexB = (float *) malloc(vertexSize);
	texB = (float *) malloc(texSize);
	normalB = (float *) malloc(normalSize);
	grid_strips = (unsigned int **) malloc(sizeof (unsigned int *) * n_strips);
\end{lstlisting}
A função começa por reserver espaço temporário na memória principal para armazenar os buffers.

\begin{lstlisting}[caption=Cálculo dos buffers para VBOs (vértices)]
	//preencher o buffer de vertices
	float *vertexAux = vertexB;
	float *texAux = texB;
	for (int x = 0; x < grid_n; x++) {
		for (int y = 0; y < grid_n; y++) {
			vertexAux[0] = grid_width * x;
			vertexAux[2] = grid_width * y;
			vertexAux[1] = this->map_h(x, y);
			vertexAux += 3;

			texAux[0] = x;
			texAux[1] = y;
			texAux += 2;
		}
	}
\end{lstlisting}
De seguida são preenchidos os buffers com as coordenadas dos vértices, e respectivas coordenadas de textura.

\begin{lstlisting}[caption=Cálculo dos buffers para VBOs (normais)]
	float *normalAux = normalB;
	for (int x = 0; x < grid_n; x++) {
		for (int y = 0; y < grid_n; y++) {
			Vertex *N = (x == 0) ? vertexFromBuffer(vertexB, x, y) : vertexFromBuffer(vertexB, x - 1, y);
			Vertex *S = (x == grid_n - 1) ? vertexFromBuffer(vertexB, x, y) : vertexFromBuffer(vertexB, x + 1, y);
			Vertex *W = (y == 0) ? vertexFromBuffer(vertexB, x, y) : vertexFromBuffer(vertexB, x, y - 1);
			Vertex *E = (y == grid_n - 1) ? vertexFromBuffer(vertexB, x, y) : vertexFromBuffer(vertexB, x, y + 1);

			Vertex *normA = new Vertex(S->x - N->x, S->y - N->y, S->z - N->z);
			Vertex *normB = new Vertex(E->x - W->x, E->y - W->y, E->z - W->z);

			Vertex *norm = new Vertex(normA->y * normB->z - normA->z * normB->y,
				normA->z * normB->x - normA->x * normB->z,
				normA->x * normB->y - normA->y * normB->y);

			norm->normalize();

			normalAux[0] = -norm->x;
			normalAux[1] = -norm->y;
			normalAux[2] = -norm->z;
			normalAux += 3;
		}
	}
\end{lstlisting}
O cálculo das normais, por depender dos vértices adjacentes, não pode ser calculado em paralelo com os restantes, por isso é aqui feito isoladamente, recorrendo a um algoritmo que estima uma aproximação à normal com base nos 4 vértices ortogonalmente adjacentes.

\begin{lstlisting}[caption=Cálculo dos buffers para VBOs (índices)]
	//preenche as strips
	for (int x = 0; x < n_strips; x++) {
		grid_strips[x] = (unsigned int *) malloc(stripSize);
		unsigned int *stripAux = grid_strips[x];
		for (int y = 0; y < grid_n; y++) {
			stripAux[0] = y + (x + 1) * grid_n;
			stripAux[1] = y + x * grid_n;
			stripAux += 2;
		}
	}
\end{lstlisting}

Finalmente, são calculados os arrays dos índices indicando a ordem pela qual os vértices devem ser desenhados. O mapa fica dividido em strips pelo que deve ser gerado um array de índices por cada strip.

\begin{lstlisting}[caption=Cálculo dos buffers para VBOs (envio dos buffers)]
	//gera e envia os buffers
	glGenBuffers(MAP_BUFF_COUNT, buffers);

	glBindBuffer(GL_ARRAY_BUFFER, buffers[MAP_VERTEX]);
	glBufferData(GL_ARRAY_BUFFER, vertexSize, vertexB, GL_STATIC_DRAW);
	glVertexPointer(3, GL_FLOAT, 0, 0);

	glBindBuffer(GL_ARRAY_BUFFER, buffers[MAP_TEX]);
	glBufferData(GL_ARRAY_BUFFER, texSize, texB, GL_STATIC_DRAW);
	glTexCoordPointer(2, GL_FLOAT, 0, 0);

	glBindBuffer(GL_ARRAY_BUFFER, buffers[MAP_NORMAL]);
	glBufferData(GL_ARRAY_BUFFER, normalSize, normalB, GL_STATIC_DRAW);
	glNormalPointer(GL_FLOAT, 0, 0);

	//liberta os buffers temporarios
	free(texB);
	if (drawNormals == false) {
		free(vertexB);
		free(normalB);
	}
}
\end{lstlisting}

Para terminar, a função envia os buffers calculados para a placa gráfica, e liberta a memória temporária alocada no início.

O render do mapa passa então a ser um ciclo simples, invocando os arrays de strips aqui calculados:

\begin{lstlisting}[caption=Render do mapa com VBOs]
void Map::render() {
	int x = 0, y = 0;

	glBindTexture(GL_TEXTURE_2D, tex_soil.gl_id);

	glMaterialfv(GL_FRONT, GL_AMBIENT, mat_amb);
	glMaterialfv(GL_FRONT, GL_AMBIENT, mat_diff);
	glMaterialfv(GL_FRONT, GL_SPECULAR, mat_spec);

	for (int x = 0; x < n_strips; x++) {
		glDrawElements(GL_TRIANGLE_STRIP, grid_n * 2, GL_UNSIGNED_INT, grid_strips[x]);

	}

	glBindTexture(GL_TEXTURE_2D, 0);
}
\end{lstlisting}
